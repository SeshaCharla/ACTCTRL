\section{Control Model and Input}
From the results of system modelling and identification, we have the non-linear
model of the system and the corresponding model parameter estimates.

\begin{equation}\label{eqn::nl_model}
    J \dot \omega + b_m \omega + C_D \omega^2 + M_f \delta v = V_{in} b_m u_\omega + V_{in}^2 (1 + \delta v) C_D u_\omega^2
\end{equation}


\begin{table}[H]
    \centering
    \begin{tabular}{c l l c}
        \hline \hline
        Parameter & Value & & $\sigma$            \\ \hline \hline
        $C_T$ & $7.2581 \times 10^{-06}$ & $N/(rad/s)^2$   & $4.4522 \times 10^{-8}$ \\
        $C_D$ & $3.6088 \times 10^{-08}$ & $N.m/(rad/s)^2$ & $1.3964 \times 10^{-9}$ \\
        $b_m$ & $0.0$                    & $N.m/(rad/s)$   & $4.6003 \times 10^{-6}$  \\
        $M_f$ & $1.3135 \times 10^{-3}$  & $N.m$           & $4.5277 \times 10^{-3}$ \\
        $J$   & $3.2238 \times 10^{-6}$   & $Kg.m^2$        & $7.0053 \times 10^{-6}$ \\
        \hline \hline
    \end{tabular}
    \caption{Summary of parameter estimates from static and small-perturbation experiments}
    \label{tab::parm_ests}
\end{table}

From parameter estimates (Table~\ref{tab::parm_ests}), it can be seen that the
estimate of total 'damping' factor $( b_m )$ is zero. Thus, the linear term of
the control input in the RHS of eqn.~\ref{eqn::nl_model} can be ignored.
Let $u = u_\omega ^2 $ be the control input to the system for which we are going
to design the feedback controller. Incorporating the above two assumptions into
eqn.~\ref{eqn::nl_model}, we have the control form of the model:
\begin{equation} \label{eqn::control_form}
    J \dot \omega + b_m \omega + C_D \omega^2 + M_f \delta v = V_{in}^2 (1 + \delta v) C_D u
\end{equation}

We have the following bounds on the control input:
\begin{align*}
    u &= u_\omega^2 = \lr{a u_p + b}^2\\
    \implies u_{min} &= \lr{a u_{p_{min}}  + b}^2 = \lr{0.0696 \times 1110 - 64.3266}^2 = 167.1694\\
    \implies u_{max} &= \lr{a u_{p_{max}}  + b}^2 = \lr{0.0696 \times 1890 - 64.3266}^2 = 4518.1789\\
\end{align*}

The goal of feedback control design for the actuator is two-fold:
\begin{enumerate}
\item Compensate for the input-uncertainities, un-modelled disturbances and
model-structure errors.
\item Make the actuator track the response of a second-order transfer function
with no over-shoot of the form:
\begin{align*}
    G_{ref}(s) &= \frac{1}{s^2 + 2 \zeta \omega_{ref} + \omega_{ref}^2}
    && \zeta = \frac{1}{\sqrt{2}} = 0.707
\end{align*}
Such that, $\omega_{ref}$ results in the maximum possible bandwidth in presence
of uncertainties mentioned above.
\end{enumerate}

To this end, two feedback control designs based on Adaptive Robust Control
Theory (ARC) are implemented, and their performances are compared:
\begin{enumerate}
    \item Direct/Indirect Adaptive Robust Controller (DIARC).
    \item ARC design with parameter estimation only for the disturbances(Disturbance ARC).
    \item Feedback linearization.
\end{enumerate}

%===============================================================================
