The Castle Creations ESC incorporates a microcontroller that employs a non-linear scaling mechanism to transform the input PWM signal's duty cycle ($u_p$) into the duty cycle of the $24,kHz$ PWM signals sent to the inverter, thereby effectively adjusting the source voltage applied to the motor\cite{kim2017electric}. This non-linear transformation aims to achieve a linear input-to-thrust relationship, departing from the quadratic one. Consequently, the PWM input to the ESC ($u_p$) undergoes filtering through a non-linear function, denoted as $g_m$, resulting in the PWM duty-cycle input to the inverter ($u_m$). This relationship can be represented as:
%===
\begin{align}\label{eqn::esc_input}
    u_m &= g_m(u_p)\quad
    V_s = u_m V_{in} \qquad u_m \in [0, 1]
\end{align}
\begin{align*}
\text{Where,}\qquad&\\
    V_s &- \text{Effective voltage to the motor (lumped)}\\
    V_{in} &- \text{Battery voltage}
\end{align*}
%===
A constant PWM switching frequency of $400 , Hz$ is utilized, and the duty cycle is scaled accordingly with this frequency. Notably, the current ESC equipped with RPM feedback capabilities operates within the range of $1110 , \mu s$ to $1890 , \mu s$. Beyond this range, the ESC switches to a constant power mode, maintaining a constant RPM.

It's worth noting that the specific parameters governing the non-linear filter are not ascertainable with the available information. Consequently, neither $u_m$ nor $u_p$ can be considered the true input for system identification in conjunction with the propeller. To circumvent this issue, we redefine the input as the motor's angular velocity with the propeller, normalized by the input voltage ($u_\omega$). We then establish a static mapping between this quantity and the PWM input to the ESC ($u_p$).
