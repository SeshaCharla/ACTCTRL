\subsection{Actuation Limits}
The actuator has hard limits beyond which the system doesn't function and
operational limits beyond which the system doesn't produce and useful output.
These are tabulated in the following:
\begin{table}[H]
    \centering
    \begin{tabular}{r c c c c}
        \hline \hline
        Limit & $u_p$ & $u_{\omega}$ & $u_{in} (=u_\omega^2)$ & $\omega$ \\ \hline \hline
        Actual Lower         & 1110 & 12.93 & 167.17 & 200.92 \\
        Operational Lower    & 1294 & 25.74 &  662.55 & 400 \\
        Operational Higher   & 1849 & 64.35 & 4140.92 & 1000\\
        Actual Higher        & 1890 & 67.23 & 4518.12 &  1044.56\\
        \hline \hline
    \end{tabular}
    \caption{Input and steady state limits of the actuator}
\end{table}


\noindent The goal of feedback control design for the actuator is two-fold:
\begin{enumerate}
\item Compensate for the input-uncertainities, un-modelled disturbances and
model-structure errors.
\item With in the operational limits, make the actuator track the response of a second-order transfer function
with no over-shoot of the form:
\begin{align*}
    G_{ref}(s) &= \frac{1}{s^2 + 2 \zeta \omega_{ref} + \omega_{ref}^2}
    && \zeta = \frac{1}{\sqrt{2}} = 0.707
\end{align*}
Such that, $\omega_{ref}$ results in the maximum possible bandwidth in presence
of uncertainties mentioned above.
\end{enumerate}
