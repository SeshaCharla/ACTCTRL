\section{Actuation limits and the reference system design}
The actuator has hard limits beyond which the system doesn't function and
operational limits beyond which the system doesn't produce and useful output.
These are tabulated in the following:
\begin{table}[H]
    \centering
    \begin{tabular}{l r r r r}
        \hline \hline
        Limit & $u_p$ & $u_{\omega}$ & $u$ & $\omega$ \\ \hline \hline
        Actual Lower         & 924  & 0     & 0        & 0\\
        Lower Start          & 1110 & 12.93 & 451.95   & 200.92 \\
        Operational Lower    & 1294 & 25.74 & 1791.08  & 400 \\
        Operational Higher   & 1849 & 64.35 & 11194.22 & 1000\\
        Actual Higher        & 1890 & 67.23 & 12218.64 &  1044.56\\
        \hline \hline
    \end{tabular}
    \caption{Input and steady state limits of the actuator}
\end{table}


% ==============================================================================

The goal of feedback control design for the actuator is two-fold:
\begin{enumerate}
\item Compensate for the input-uncertainities, un-modelled disturbances and
model-structure errors.
\item With in the operational limits, make the actuator track the response of a second-order transfer function
with no over-shoot of the form:
\begin{align*}
    G_{r}(s) &= \frac{\omega_{r}^2}{s^2 + 2s \zeta \omega_{r} + \omega_{r}^2}
    && \zeta = \frac{1}{\sqrt{2}} = 0.707
\end{align*}
Such that, $\omega_{r}$ results in the maximum possible bandwidth in presence
of uncertainties mentioned above.
\end{enumerate}

The above reference system will also introduce a limit on the maximum rate of
change of angular velocity. We have,
\begin{align*}
    &\omega(s) = \frac{\omega_{r}^2}{s^2 + 2 s \zeta \omega_{r} + \omega_{r}^2}\\
    %===
    &\text{Let,} \quad \alpha = \frac{d \omega }{dt}
    \\
    %===
    &\implies \alpha(s) = s \omega(s) =  \frac{s\omega_{r}^2}{s^2 + 2 s \zeta \omega_{r} + \omega_{r}^2}
    \\
    %===
    &\implies \norm{\alpha(j\omega)} = \norm{\frac{j \omega \omega_{r}^2}{{j \omega}^2 + 2 j \omega  \zeta \omega_{r} + \omega_{r}^2}}
    %---
    = \frac{\omega \omega_r^2}{\norm{-\omega^2 + 2 j \omega  \zeta \omega_{r} + \omega_{r}^2}}
    %---
    = \frac{\omega \omega_r^2}{\sqrt{\lr{\omega_r^2 - \omega^2}^2 + 4 \zeta^2 \omega_r^2 \omega^2}}
    \\
    %===
    &\text{as, }\quad \zeta = \frac{1}{\sqrt{2}},
    \qquad
    %===
    \norm{\alpha(j \omega)}^2 = \frac{\omega^2 \omega_r^4}{\omega^4 + \omega_r^4}
\end{align*}
% ==============================================================================

Let, $\Omega_r = \omega_r^2$, $\Omega = \omega^2$ and $M(\Omega) =
\norm{\alpha(j\omega)}^2$.
Thus, the maximum rate of change happens at:
\begin{align*}
    &\frac{d M(\Omega)}{d \Omega} = \frac{d}{d \Omega} \lr{\frac{\Omega \Omega_r^2}{\Omega^2 + \Omega_r^2}} = 0
    \quad
    %===
    \implies \frac{\Omega_r^2 \lr{\Omega^2 + \Omega_r^2} - 2 \Omega \lr{\Omega \Omega_r^2}}{\lr{\Omega^2 + \Omega_r^2}^2} = 0
    \quad
    %===
    \implies \Omega = \Omega_r
\end{align*}
Thus,
\begin{align*}
    \max \lr{{ \norm{\alpha(j \omega)}^2 }} &= M(\Omega_r) = \frac{\Omega_r^3}{2 \Omega_r^2} = \frac{1}{2} \omega_r^2
\end{align*}
\begin{equation}\label{eqn::max_bandwidth}
    \therefore \; \dot \omega_{max} = \frac{1}{\sqrt{2}} \omega_r
\end{equation}

% ==============================================================================
% ==============================================================================

\subsection{Limitations on $\dot \omega$}

The input saturation imposes limitations on the maximum and minimum rate of
change of the angular velocity $(\omega)$. From the dynamic model (\ref{eqn::ctrl_form}), removing the zero parameters
$(b_m)$ and uncertainities $(\Delta)$, we have:

\begin{align*}
    \dot \omega &= -C_{D_J} \omega^2 - M_{f_J} + V_{in} u
\end{align*}

From the above equation, we have the following bounds on the rate of change of
$\omega$, with in the operational limits [400 rad/s, 1000 rad/s].

\begin{enumerate}
    \item Maximum rate of increase:
    \begin{align*}
        \dot \omega^{+}_{max} &= -C_{D_J} (\omega_{min})^2 - M_{f_J}+ V_{in} u_{max}\\
                              &= -C_{D_J} (400)^2 - M_{f_J} + V_{in} (12218.64)
                               =  189877.66 \; rad/s
                               = 1.899e5 \; rad/s
    \end{align*}

    \item Minimum rate of increase:
    \begin{align*}
        \dot \omega^{+}_{min} &= -C_{D_J} (\omega_{max})^2 - M_{f_J}+ V_{in} u_{max}\\
                              &= -C_{D_J} (1000)^2 - M_{f_J} + V_{in} (12218.64)
                               = 189459.03 \; rad/s = 1.894e5 \; rad/s
    \end{align*}

    \item Maximum rate of decrease:
    \begin{align*}
        \dot \omega^{-}_{max} &= -C_{D_J} \omega_{min}^2 - M_{f_J} + V_{in} u_{min}\\
                              &= -C_{D_J} 1000^2 - M_{f_J} + V_{in} \times 0
                               = -418.63 \; rad/s
    \end{align*}

    \item Minimum rate of decrease:
    \begin{align*}
        \dot \omega^{-}_{min} &= -C_{D_J} \omega_{min}^2 - M_{f_J} + V_{in} u_{min}\\
                              &= -C_{D_J} 400^2 - M_{f_J} + V_{in} \times 0  = -411.91 \; rad/s
    \end{align*}
\end{enumerate}

% ==============================================================================

For the motor-propeller system to perfectly track the response of a linear
second order system within the input constraints, the fastest rate of change of
reference system state should be less than the minimum rate of decrease in the
actual system, i.e.,

\begin{align}
    \abs{\dot \omega_{max}} &\leq \abs{\dot \omega^{-}_{min}} \label{eqn::dot_omega_max}\\
    \implies \frac{1}{\sqrt{2}} \omega_r &\leq 411.91 \; rad/s
    [\because \ref{eqn::max_bandwidth}]\\
    \therefore \omega_r &\leq 582.53
    \label{eqn::omega_r_lim}
\end{align}

% ==============================================================================
% ==============================================================================

\subsection{Sampling Limitations}
The sampling due to the implimentation of feedback would introduce it's own
limitations on the maximum bandwidth of the reference system, namely, the
Nyquist frequency $(n_f)$ of sampling. The approximate inequality would be:
\begin{align}
    \omega_r < 2 \pi n_f
    \label{eqn::wr_lim_nf}
\end{align}

For the current set-up, $n_f = 250/2 = 125 \; Hz$. Thus,
\begin{align}
    \omega_r < 2\pi \times 125 = 785.4 \; rad/s
\end{align}

The above limit is less restrictive than the limit on the maximum rate of change
due to the passive dissipative forces on the actuator.


\bigskip \itbf{Note:} The above inequality (\ref{eqn::wr_lim_nf}), we can fliped and
combined with (\ref{eqn::dot_omega_max}) for using it as a design limitation
for the samplig system. Thus, the minimum sampling the requirement for obtating
the maximum performace from the system as:
\begin{align}
    n_f > \frac{\dot \omega^{-}_{min}}{2 \pi}
\end{align}


% ==============================================================================
% ==============================================================================

\subsection{Reference system design}
From the above inequalites on $\omega_r$, (\ref{eqn::wr_lim_nf}),
(\ref{eqn::omega_r_lim}), a conservative choice or $\omega_r$ is used for the
reference system design.

\begin{align}
    \omega_r &= 550 \; rad/s\\
    \zeta &= \frac{1}{\sqrt{2}}
\end{align}

Thus,
\begin{align}
    G_r (s) &= \frac{302500}{s^2 + 777.82 s + 302500}
\end{align}
