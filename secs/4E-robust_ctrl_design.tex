\subsection{Robust feedback control design}

Adding model uncertainties to the error dynamics (\ref{eqn::error_feedback_dyn}):
\begin{align}
    \dot z + bk z &=  \tilde \phi^T \theta + bu_{s2} - \lrb{ \phi_{du}^T \tilde \theta_b - d}
    \label{eqn::error_feedback_dyn_uncertain}
\end{align}

The left side of (\ref{eqn::error_feedback_dyn_uncertain}) represents the
stable nominal closed-loop dynamics. The terms in the brackets of the right
hand side represents the effect of all model uncertainties. Though these terms
are unknown, but they are bounded above with some known function h(x, t). The
goal of the robust feedback is to account for the uncertainties during the
transients and the steady state of parameter adaptation. Thus, $u_{s2}$ is
synthesized to satisfy the following two conditions:
\begin{align}
    C2:& \qquad z \lrb{-\phi_{du}^T \tilde \theta_b + d + b u_{s2}} \leq \varepsilon \\
    C3:& \qquad z b u_{s2} \leq 0
\end{align}

%===============================================================================

The function $h(x, t)$ is calculated as follows:

\begin{align}
    \abs{\phi_{du}^T \tilde \theta_b - d} &\leq h\\
    \implies h(x_d, t) &= \abs{\phi_{du}^T(x_d, t)}  \tilde \theta_{bM} + 2 d_M \qquad \lrb{\because d \leq 2d_M }
\end{align}

The robust feedback control input $u_{s2}$ is thus designed to satisfy the conditions $C2$ and $C3$ as follows:
\begin{align}
    u_{s_2} &= -\frac{1}{\hat b_{min}} S(h \sign(z)) \label{eqn::robust_u}\\
    %===
    \text{Where,  } \qquad
    S(h \sign(z)) &= h \sat \lr{ \frac{h}{4 \varepsilon} z} &[\text{Smoothing function}]\\
    %===
    \sat (x) &= \begin{cases}
        x  & \text{if  } \abs{x} \leq   1\\
        \sign(x) &  \text{otherwise}
    \end{cases}
\end{align}

The above defined smoothing function, $S(.)$, has the following properties:

\begin{itemize}
\item[$P_3$:]
\begin{align}
    z S(h \sign(z)) = z h \sat \lr{\frac{h}{4 \varepsilon} z} \geq 0
\end{align}
$\because$ $z$ and $\sat \lr{\frac{h}{4 \varepsilon} z}$ have the same sign.

\item[$P_4$:]
\begin{align}
    z \left[h \sign(z) - S(h\sign(z)) \right] \text{  is bounded.}
\end{align}
\itbf{proof:}
\begin{align*}
    z \left[h \sign(z) - S(h\sign(z)) \right] &= z \left[h \sign(z) - h \sat \lr{\frac{h}{4 \varepsilon} z} \right]
\end{align*}

\begin{enumerate}
\item Outside the boundary layer, i.e., $\lr{\abs{z} > \frac{4 \varepsilon}{h}
}$:
\begin{align*}
    zh \sign(s) - z h \sat \lr{\frac{h}{4 \varepsilon} z} &= h \abs{z} - h\abs{z} = 0
\end{align*}

\item Inside the boundary layer, i.e., $\lr{\abs{z} < \frac{4 \varepsilon}{h}}$
\begin{align*}
    zh \sign(z) - z h \sat \lr{\frac{h}{4 \varepsilon} z} &= h \abs{z} - \frac{h^2}{4 \varepsilon} z^2\\
    &= \varepsilon - \left[ \frac{1}{2 \sqrt{\varepsilon}} h \abs{z} - \sqrt{\varepsilon} \right]^2\\
    &\leq \varepsilon\\
\end{align*}
\end{enumerate}
\end{itemize}

The rubust control input definition (\ref{eqn::robust_u}) satisfies the
conditions $C2$ and $C3$ as following:
\begin{enumerate}
    \item[C2:]
    \begin{align*}
        z b u_{s2} &= -z \lr{\frac{b}{\hat b_{min}}} S(h \sign(z)) < 0
        \qquad \lrb{\because \frac{b}{b_{min}} > 1} \\
    \end{align*}
As the input gain doesn't change sign, the above inequality is always satisfied.

    \item[C3:]
    \begin{align*}
        z \lrb{-\phi_{du}^T \tilde \theta_b + d + b u_{s2}} &= z \lrb{-\phi_{du}^T \tilde \theta_b + d - \delta b S(h \sign(z))} \qquad \lrb{\text{where, } \quad \delta b = \frac{b}{\hat b_{min}} \geq 1 }\\
        %===
        &\leq z \lrb{h(x_d, t) - \delta b S(h \sign(z))}\\
        %===
        &\leq z \lrb{h(x_d, t) - S(h \sign(z))} \leq \varepsilon
        \qquad [\because \text{Property } P_4]
    \end{align*}

\end{enumerate}
