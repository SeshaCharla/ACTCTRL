In this section, a discontinuous projection based direct adaptive robust control
is implemented. Though the parameter convergence and tracking performance is
slightly lower than the  more advanced designs such as IARC and DIARC, DARC has
the advantage of being least computationally complex among them, resulting in
faster sampling times in resource limited scenarios.


\subsection{Parametric model and parameter bounds for ARC design}

The parameter bounds are obtained from the variance of estimates and the
maximum value of $\delta v$. For the current design $\pm 2\sigma$ bounds are
used for each of the $\theta$'s. From experimental observations, the maximum
supply voltage is about $16\, V$ and minimum supply voltage is close to $15\,
V$. The average would be in the middle ($15.5\, V$). The physics dictates that
all the physical parameters have a non-negative value. Also, the minimum value
of $J$ is assumed to be $1/10$ of its nominal value instead of zero. This is to
ensure that the inertia estimate would never be zero and with-in the reasonable
limits. The following table gives the minimum, maximum and nominal values of the parameters:

\begin{table}[H]
    \centering
    \begin{tabular}{r l r r r}
        \hline \hline
        Parameter & Units &Nominal & Minimum & Maximum \\ \hline \hline
        $C_T$                   &
        $N/(rad/s)^2$           &
        $7.2581e-6$             &
        $7.1690e-6$             &
        $7.3471e-6$
        \\
        $C_D$                   &
        $N.m/(rad/s)^2$         &
        $3.6088e-8$             &
        $3.3295e-8$             &
        $3.8881e-8$

        \\
        $b_m$                    &
        $N.m/(rad/s)$            &
        $0$                      &
        $0$                      &
        $9.2006e-6$
        \\
        $M_f$                    &
        $N.m$                    &
        $1.3135e{-3}$            &
        $0$                      &
        $0.0104$
        \\
        $J$                      &
        $Kg.m^2$                 &
        $3.2238e{-6}$            &
        $3.2238e{-7}$            &
        $1.7234e-5$
        \\
        $V_{in}$                 &
        $V$                      &
        $15.5$                   &
        $15$                     &
        $16$
        \\
        $\delta v$               &
                                 &
        $0$                      &
        $-0.2$                   &
        $0.2$
        \\
        \hline \hline
    \end{tabular}
    \caption{Summary of parameter bounds}
    \label{tab::parm_bounds}
\end{table}

\subsubsection{Parametric Model}
It can be noted that the variance of $\hat J$ is substantial as compared to its
actual value. This would propagate into the all the parameters if it is divided
on all sides. Instead, we normalize all the coefficients with the controller
gain to minimize the propagation of variance to the parameters. Thus,
eqn.~\ref{eqn::control_form} becomes:

\begin{align*}
    \lr{\frac{J}{V_{in}^2 C_D}} \dot \omega + \lr{\frac{b_m}{V_{in}^2 C_D}} \omega + \lr{\frac{1}{V_{in}^2}} \omega^2  &= u + \delta v \lr{u - \frac{M_f}{V_{in}^2 C_D}}\\
    %===
    \text{Let, } \qquad &\\
    \theta_1 = \lr{\frac{J}{V_{in}^2 C_D}} \qquad
    \theta_2 &= \lr{\frac{b_m}{V_{in}^2 C_D}}\qquad
    \theta_3 = \lr{\frac{1}{V_{in}^2}}\\
    d(t) &= \delta v \lr{u - \frac{M_f}{V_{in}^2 C_D}}\\
    %===
    \implies \theta_1 \dot \omega + \theta_2 \omega + \theta_3 \omega^2 &= u + d(t)
\end{align*}
\begin{equation}\label{eqn::almst_ctrl_mdl}
    \theta_1 \dot \omega = u - \theta_2 \omega - \theta_3 \omega^2 + d(t)
\end{equation}

Let, $\omega_d(t)$ be the desired trajectory that the system needs to track
(output of $G_{ref}(s)$). Thus, the tracking error:
\begin{align*}
    s &= \omega - \omega_d \quad \implies \dot s = \dot \omega - \dot \omega_d
\end{align*}
Dividing the disturbance into high-frequency component and a slowly varying
component, let, $$d(t) = d_0 + \Delta(t)$$
Thus, subtracting $\theta_1 \omega_d$ on both sides of eqn.~\ref{eqn::almst_ctrl_mdl}.
\begin{align*}
    \theta_1 (\dot \omega - \dot \omega_d) &= u - \theta_1 \omega_d - \theta_2 \omega - \theta_3 \omega^2 + d_0 + \Delta(t)\\
    \text{Let,} \qquad\\
    \pmb \theta &= \bm{\theta_1 & \theta_2 & \theta_3 & d_0}^T\\
    \pmb \phi &= \bm{-\omega_d & -\omega & - \omega^2 & 1}^T\\
\end{align*}
Thus we have the model for tracking error dynamics:
\begin{equation}\label{eqn::error_dyn}
    \theta_1 \dot s = u + \pmb \phi^T \pmb \theta + \Delta(t)
\end{equation}


\subsubsection{Parameter Bounds}
The parameter bounds for eqn.~\ref{eqn::error_dyn} are obtained from the
parameter bounds of estimated parameters. Also,
\begin{align*}
    \abs{d(t)} &\leq d_M \implies \abs{d_0} \leq d_M \text{  and  } \abs{\Delta} \leq 2d_M = \Delta_M\\
    \text{From Table~\ref{tab::parm_bounds}} \qquad &\\
    d_M &= \delta v_{max} \lr{u_{max} - \frac{M_{f_{min}}}{V_{in_{max}}^2 C_{D_{max}}}} \approx 900
\end{align*}
Similarly, we have
\begin{table}[H]
    \centering
    \begin{tabular}{r l r r r}
        \hline \hline
        Parameter & Equation & Nominal & Minimum & Maximum\\ \hline \hline
        $\theta_1$ &
        $= \lr{\frac{J}{V_{in}^2 C_D}}$ &
        $3.7183e-1$ &
        $3.2388e-2$ &
        $2.3005$
        \\
        $\theta_2$ &
        $= \lr{\frac{b_m}{V_{in}^2 C_D}}$ &
        $0$ &
        $0$ &
        $1.0794$
        \\
        $\theta_3$ &
        $= \lr{\frac{1}{V_{in}^2}}$ &
        $4.1623e-3$ &
        $3.9062e-3$ &
        $4.4444e-3$
        \\
        \hline \hline
    \end{tabular}
    \caption{Parameter Bounds}
    \label{tab::parm_lims}
\end{table}

\subsection{Control Input Design}
Let, the total control input be the sum of the following inputs:
\begin{align*}
    u &= u_a + u_s \qquad u_s = u_{s_1} + u_{s_2}
\end{align*}
%===============================================================================
\subsubsection{Model compensation input design with parameter adaption ($u_a$)}
Let, From eqn.~\ref{eqn::error_dyn}, we chose $u_a$ such that it compensates for
the error dynamics, i.e.,
\begin{align*}
    u_a &= - \pmb \phi^T  \hat{\pmb{\theta}} \\
    \text{where, } \qquad &\\
    \dot{\hat{\pmb{\theta}}} &= Proj_{\hat{\pmb{\theta}}}\lr{\Gamma \pmb \phi s}\\
\end{align*}
Where $\Gamma$ is a positive definite, diagonal adaption rate matrix.

We define, \itbf{Discontinuous projection mapping} $Proj()$ for a vector as follows:
\begin{align*}
    Proj_{\hat{\pmb\theta}} (\bullet) &= \bm{Proj_{\hat \theta_1}(\bullet_1), &
                                        Proj_{\hat \theta_2}(\bullet_2), &
                                        \hdots, &
                                        Proj_{\hat \theta_n}(\bullet_n)}\\
    Proj_{\hat \theta_i}(\bullet_i) &= \begin{cases}
        0 & \text{if } \begin{cases}
                        \hat \theta_i &= \hat \theta_{i_{min}} \text{  and  } \bullet_i < 0\\
                        \hat \theta_i &= \hat \theta_{i_{max}} \text{  and  } \bullet_i > 0\\
                       \end{cases}\\
        \bullet_i & \text{otherwise}
    \end{cases}
\end{align*}
The above mapping has the following properties:
\begin{itemize}
    \item[$P_1$:] $\quad \hat{\pmb \theta} \in \bar \Omega_{\theta}=\{\hat{\pmb \theta} : \theta_{i_{min}} \leq \hat \theta_i \leq \theta_{i_{max}} \; \forall i\} \; \forall t$

    \item[$P_2$:] $\quad \tilde{\pmb{\theta}} \left[ \Gamma^{-1}
    Proj_{\hat{\pmb\theta}} \lr{\Gamma \pmb \phi s} - \phi s \right] \leq 0 \:
    \forall t \qquad [\tilde{\pmb{\theta}} = \hat{\pmb\theta} - \pmb \theta]$
\end{itemize}

%===============================================================================

\subsubsection{Robust feedback control input design}
Substituting $u_a$ in eqn.~\ref{eqn::error_dyn}:
\begin{align*}
    \theta_1 \dot s &= -\pmb \phi^T \hat{\pmb \theta} + \pmb \phi^T \pmb \theta + \Delta + u_{s_1} + u_{s_2}\\
    \text{Let,  } u_{s_1} &= -k_p s & & [\text{Proportional Feedback}]\\
    \text{and  } \tilde{\pmb\theta} &= \hat{\pmb \theta} - \pmb \theta\\
    \implies \theta_1 \dot s + k_p s &= u_{s_2} - \pmb \phi^T \tilde{\pmb \theta} + \Delta\\
\end{align*}
Let $- \pmb \phi^T \tilde{\pmb \theta} + \Delta$ be bounded by $h(\omega, t)$,
i.e.,
\begin{align*}
    h(\omega, t) &\geq   - \pmb \phi^T \tilde{\pmb \theta} + \Delta  \quad \forall t\\
    \implies h &= \abs{\pmb \phi^T} \tilde{\pmb \theta}_M + \abs{\Delta}\\
               &= \abs{\pmb \phi^T} \tilde{\pmb \theta}_M + \Delta_M \qquad
                \tilde{\pmb \theta}_M = \pmb \theta_{max} - \pmb \theta_{min}\\
\end{align*}

Use sliding mode control law for $u_{s_2}$:
\begin{align*}
    u_{s_2} &= S(h \sign(s))\\
    \text{Where,  } \qquad &\\
    S(h \sign(s)) &= h \sat \lr{ \frac{h}{4 \varepsilon} s} &[\text{Smoothing function}]\\
    \sat (x) &= \begin{cases}
        x  & \text{if  } \abs{x} \leq   1\\
        \sign(x) &  \text{otherwise}
    \end{cases}
\end{align*}
The above defined smoothing function, $S(.)$, has the following properties:

\begin{itemize}
\item[$P_3$:] $s S(h \sign(s)) = s h \sat \lr{\frac{h}{4 \varepsilon} s} \geq
0 \qquad $ $\because$ $s$ and $\sat \lr{\frac{h}{4 \varepsilon} s}$ have the
same sign.

\item[$P_4$:] $s \left[h \sign(s) - S(h\sign(s)) \right]$ is bounded.\\
\itbf{proof:}
\begin{align*}
    s \left[h \sign(s) - S(h\sign(s)) \right] &= s \left[h \sign(s) - h \sat \lr{\frac{h}{4 \varepsilon} s} \right]
\end{align*}
\begin{enumerate}
\item Outside the boundary layer, i.e., $\lr{\abs{s} > \frac{4 \varepsilon}{h}
}$:
\begin{align*}
    sh \sign(s) - s h \sat \lr{\frac{h}{4 \varepsilon} s} &= h \abs{s} - h\abs{s} = 0
\end{align*}

\item Inside the boundary layer, i.e., $\lr{\abs{s} < \frac{4 \varepsilon}{h}}$
\begin{align*}
    sh \sign(s) - s h \sat \lr{\frac{h}{4 \varepsilon} s} &= h \abs{s} - \frac{h^2}{4 \varepsilon} s^2\\
    &= \varepsilon - \left[ \frac{1}{2 \sqrt{\varepsilon}} h \abs{s} - \sqrt{\varepsilon} \right]^2\\
    &\leq \varepsilon\\
    s \left[h \sign(s) - S(h\sign(s)) \right] &\leq \varepsilon
\end{align*}
\end{enumerate}
\end{itemize}

Thus, we have the DARC control law:
\begin{align*}
    u &= u_a + u_{s_1} + u_{s_2}\\
    u_a &= - \pmb \phi^T \hat{\pmb \theta} \qquad  \qquad
    \dot{\hat{\pmb \theta}} = Proj_{\hat{\pmb \theta}} \lr{\Gamma \pmb \phi s}\\
    u_{s_1} &= -k_p s\\
    u_{s_2} &= S(h \sign(s)) = h \sat \lr{\frac{h}{4 \varepsilon} s}
\end{align*}

\subsection{Stability and tracking performance}
\subsubsection{Case 1: $\Delta = 0$}
When $\Delta = 0$ consider the following Lyapunov function:
\begin{align*}
    V &= \frac{1}{2} \theta_1 s^2 + \frac{1}{2} \tilde{\pmb \theta}^T \Gamma^{-1} \tilde{\pmb \theta}\\
    %===
    \dot V &= s \theta_1 + \tilde{\pmb \theta} \Gamma^{-1} \underbrace{\dot{\tilde{\pmb \theta}}}_{=\dot{\hat{\pmb \theta}}}\\
    %===
    &= s \lr{ -k_p s + \underbrace{u_{s_2}}_{=0} - \tilde{\pmb \theta} \pmb \phi } + \tilde{\pmb \theta} \Gamma^{-1} Proj_{\hat{\pmb \theta}} \lr{\Gamma \Phi s}\\
    %===
    &= -k_p s^2 + \underbrace{\tilde{\pmb \theta}^T \lr{-\pmb \phi s + \Gamma^{-1}Proj_{\hat{\pmb \theta}} \lr{\Gamma \Phi s}}}_{\leq 0} \qquad [P_2]\\
    %===
    \implies \dot V \leq -k_p s^2 \leq 0
\end{align*}
Thus,
\begin{itemize}
    \item $\dot V$ is negative semi-definite.
    \item $V$ has a lower bound $(\geq 0)$.
    \item $\dot V$ is uniform continuous $(\ddot V \leq -k_p s \dot s)$.
    \item $\therefore$ For a bounded $\omega_d$, $s \in L_\infty$, $\tilde{\pmb
    \theta} \in L_{\infty}$ and $\dot s \in L_\infty$, $\ddot V \in L_\infty$.
\end{itemize}

Hence, the parameter estimates and tracking errors are bounded and the system is
stable.

\bigskip

Also, since $V$ is non-decreasing and positive definite,
\begin{align*}
    V(0) - V(t) &\leq V(0) \quad
    \implies -\int_0^t V d \tau \leq V(0)\\
    \implies \int_0^t k_p s^2 d \tau &\leq V(0)\\
    \implies \sqrt{\int_0^t \norm{s}^2 d \tau } &\leq \sqrt{\frac{2 V(0)}{k_p}} \leq \infty\\
    \therefore s &\in L_2
\end{align*}
From Barbalat's lemma, we have,
\begin{align*}
    \dot V \rightarrow 0 \quad s \rightarrow 0 \quad \text{as} \quad t \rightarrow 0
\end{align*}
Thus the parameter estimates are bounded, and the tracking error asymptotically
goes to zero as  $t \rightarrow 0$.


\subsubsection{Case 2: $\Delta \neq 0$}
Consider the following Lyapunov function:
\begin{align*}
    V &= \frac{1}{2} \theta_1 s^2\\
    \implies \dot V &= s \theta_1 \dot s = s \lr{-k_p s + u_{s_2} - \pmb \phi^T \tilde{\pmb \theta} + \Delta }\\
    %===
    &= -k_p s^2 + s \lr{-\pmb \phi^T \tilde{\pmb \theta} + \Delta - S(h \sign(s))}\\
    &\leq -k_p s^2 + s \lr{\abs{\pmb \phi^T} \tilde{\pmb \theta}_M + \abs{\Delta} - S(h \sign(s))}
    = -k_p s^2 + \underbrace{s \lr{h\sign(s)- S(h \sign(s))}}_{\leq \varepsilon} \quad [\because P_4] \\
    \implies \dot V &\leq -k_p s^2 + \varepsilon = - \frac{2 k_p}{\theta_1} V + \varepsilon
\end{align*}
By applying comparison lemma, the upper bound can be found from the forced
response of the following ODE:
\begin{align*}
    \dot V + \frac{2 k_p}{\theta_1} V &= \varepsilon\\
    \implies sV - V(0) + \frac{2 k_p}{\theta_1} V &= \varepsilon\\
    \implies V &= \frac{V(0)}{s + \frac{2 k_p}{\theta_1}} + \frac{\varepsilon}{\frac{2 k_p}{\theta_1}}\\
    \implies V(t) &= V(0)e^{\frac{2 k_p}{\theta_1} t} + \int_0^t{e^{\frac{2 k_p}{\theta_1} (t-\tau)} \varepsilon(\tau) d\tau}\\
    \implies \frac{1}{2} \theta_1 s^2 &= \frac{1}{2} \theta_1 s(0)^2 e^{\frac{2 k_p}{\theta_1} t} + \int_0^t{e^{\frac{2 k_p}{\theta_1} (t-\tau)} \varepsilon(\tau) d\tau}\\
    %===
    \implies \abs{s(t)}^2 &= s(0)^2 e^{\frac{2 k_p}{\theta_1} t} +\frac{1}{\frac{1}{2} \theta_1 } \int_0^t{e^{\frac{2 k_p}{\theta_1} (t-\tau)} \varepsilon(\tau) d\tau}\\
    &\leq s(0)^2 e^{\frac{2 k_p}{\theta_1} t} + \frac{\varepsilon_M}{k_p} \left[1 -  e^{\frac{2 k_p}{\theta_1} t} \right]\\
    %==
\end{align*}
This the system is stable, and the transient response is bounded that
exponentially converges to $\frac{\varepsilon}{k_p}$.

\input{secs/6_3-tunning.tex}
